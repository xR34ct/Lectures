\documentclass{article}

\usepackage[swedish]{babel}
\usepackage[utf8]{inputenc}
\usepackage{amsmath}
\usepackage{hyperref}
\usepackage{underscore}
\usepackage{tikz}
\usepackage{tkz-euclide}

\hypersetup{
	pdfauthor={Adnan Avdagic},
	pdftitle={TATA69 Föreläsningar},
	colorlinks}

\begin{document}
\title{TATA69 Föreläsningar}
\author{Adnan Avdagic\\
	Linköpings Universitet\\
	\texttt{adnan@avdagic.net}}
\date{\today}
\maketitle

\newpage


\section{Föreläsning 2}
\subsection{Gränsvärden för flervarre}

\paragraph{Exempel 1} \flushleft
\begin{equation} \label{eq:1}
	f(x,y) = \frac{\sin(x^4+y^2)}{x^4+y^2}
\end{equation}

Funktionen \eqref{eq:1} är ej i definerad i origo. \newline
Vad händer då (x,y) närmar sig (0,0)?

$$\lim_{x,y \rightarrow 0,0} \frac{\sin(x^4+y^2)}{x^4+y^2}$$

//sätt $t= \displaystyle x^4+y^2$, ${t \rightarrow 0}$ då ${(x,y) \rightarrow (0,0)}$// \newline
då fås \(\displaystyle \lim_{t \rightarrow 0} \frac{\sin t}{t} = 1,\) (standard gränsvärde) \newline

\paragraph{Exempel 2} \flushleft
\begin{equation} \label{eq:2}
	f(x,y) = \frac{x^3+xy}{x^2+y^2}
\end{equation}

Funktionen \eqref{eq:2} är ej i definerad i origo.\newline

Gå mot origo via x-axeln (där $y=0$)
$$f(x,0) = \frac{x^3+0*x}{x^2+0^2} = \frac{x^3}{x^2} =  {x \rightarrow 0} \text{ då } {x \rightarrow 0}$$

Gå mot origo via y-axeln (där $x=0$)
$$f(0,y) = \frac{0^3+0*y}{0^2+y^2} = \frac{0}{y^2} =  {0 \rightarrow 0} \text{ då } {y \rightarrow 0}$$

Gå mot origo längs $y=x$
$$f(x,x) = \frac{x^3+x*x}{x^2+x^2} = {\frac{x+1}{2} \rightarrow \frac{1}{2}} \text{ då } {x \rightarrow 0}$$

Olika värden från olika riktningar \newline
Innanför varje liten cirkel kring origo har f värden nära 0 och nära $\frac{1}{2}$.
Vi säger därför att gränsvärde ej existerar. Se \ref{fig1}

\begin{figure}[ht] 
\begin{tikzpicture}
   \tkzInit[xmax=5,ymax=5,xmin=-5,ymin=-5]
   \tkzAxeXY
   \draw[red,thick] (-5,0) -- (5,0);
   \draw[blue,thick] (0,-5) -- (0,5);
   \draw[green,thick] (-5,-2) -- (5,3);
   \node[above,red] at (-4,0) {$f=x$};
   \node[right,blue] at (0,5) {$f=0$};
   \node[left,green] at (3,3) {$f=\frac{x+1}{2}$};
   \tkzDefPoint(0,0){O}
   \tkzDefPoint(0.2,0.2){A}
   \tkzDrawCircle(O,A)
  \end{tikzpicture}
  \caption{Graf i 2D} \label{fig1}
\end{figure}

\newpage
\paragraph{Definition}
\paragraph{} \flushleft
Funktionen $\bar{f}$ av typ $\mathbb{R}^n \rightarrow \mathbb{R}^m$ \newline 
Har gränsvärdet $\bar{b} \in \mathbb{R}^m$ då $\bar{x} \rightarrow \bar{a} \in \mathbb{R}^n$ om $\forall \epsilon >0 \quad \exists \delta >0$ så att $| \bar{f}(x)-\bar{b} | < \epsilon$ om $0$

\end{document}